\documentclass{svproc} % Plantilla oficial de Springer
\usepackage{graphicx}   % Para incluir imágenes
\usepackage{url}        % Para añadir enlaces

\begin{document}

\title{Sistema Inteligente para la Predicciónasdasdas del Nivel de Congestión del Tráfico}
\author{Abel Albuez Sánchez, Juan Camilo Torres Peña y Daniel Ríos Caro}
\institute{Pontificia Universidad Javeriana, Bogotá, Colombia}

\maketitle

\begin{abstract}
El incremento del tráfico en entornos urbanos ha generado problemas
de movilidad y eficiencia vial. Este proyecto propone un sistema basado en Inteligencia Artificial (IA) 
para predecir el nivel de congestión del tráfico, utilizando Redes Neuronales, Algoritmos Evolutivos, 
Lógica Difusa y Aprendizaje de Máquina. Este modelo analizará datos históricos, identificando patrones de congestión 
para optimizar la movilidad urbana. Debido a la ausencia de datos en tiempo real, 
la predicción se basará en tendencias pasadas, permitiendo anticipar escenarios de tráfico
 bajo distintos contextos temporales. La validación del modelo se realizará mediante métricas de clasificación, incluyendo ROC,
  Sensibilidad, Especificidad, F1 y Precisión, asegurando la fiabilidad de las predicciones.
   Este estudio contribuirá a la planificación y optimización del tráfico, proporcionando una herramienta de análisis para la toma de decisiones en movilidad urbana.
\end{abstract}

\noindent\textbf{Palabras Clave:} Predicción del Tráfico, Inteligencia Artificial, Redes Neuronales, Lógica Difusa, Algoritmos Evolutivos, Ciudades Inteligentes, Optimización del Tráfico.

\section{Motivación y Contextualización del Problema}
La congestión vehicular es uno de los problemas urbanos más críticos en el mundo
moderno. En ciudades altamente pobladas, el tráfico no solo afecta la movilidad de los
ciudadanos, sino que también impacta negativamente la economía, la salud y el medio
ambiente.

\subsection{Datos clave sobre la congestión del tráfico}
\begin{itemize}
    \item En EE.UU., los conductores pierden en promedio 54 horas al año debido a la
    congestión, generando un costo anual de 166 mil millones de dólares en
    productividad perdida \cite{CEPAL}.
    \item En México, la congestión vehicular representa pérdidas de 94 mil millones de
    pesos anuales, equivalente al 3.5\% del PIB regional \cite{IMCO}.
    \item Según la Organización Mundial de la Salud (OMS), la contaminación causada por
    el tráfico contribuye a 7 millones de muertes prematuras al año debido a
    enfermedades respiratorias y cardiovasculares.
\end{itemize}

La necesidad de desarrollar un sistema inteligente que pueda predecir los niveles de
congestión del tráfico se ha vuelto más relevante con el aumento del parque vehicular y
la urbanización acelerada.

\section{Estado del Arte: Aplicaciones de IA en la Predicción del Tráfico}
El uso de Inteligencia Artificial (IA) ha revolucionado la gestión del tráfico en ciudades
inteligentes. Diversas técnicas han demostrado mejorar la predicción del nivel de
congestión, optimizando la movilidad y reduciendo tiempos de espera en el tráfico. A
continuación, se presentan algunos enfoques recientes:

\subsection{Redes Neuronales Artificiales (ANNs) en Predicción del Tráfico}
Los modelos de aprendizaje profundo (Deep Learning) han alcanzado una
precisión de 85\% en la predicción del nivel de congestión, superando métodos
estadísticos tradicionales \cite{Goenawan2024}.

\subsection{Algoritmos Evolutivos en la Optimización del Tráfico}
En Los Ángeles, los algoritmos evolutivos han logrado reducir la congestión
vehicular en un 30\%, optimizando la sincronización de semáforos y las rutas
alternativas \cite{SmartMobility2024}.

\subsection{Lógica Difusa en Sistemas de Control de Tráfico}
Se ha demostrado que la Lógica Difusa aplicada a semáforos inteligentes puede
reducir los tiempos de espera en intersecciones en un 25\%, adaptando los tiempos
de luz verde en función del tráfico real \cite{TransportationScience2024}.

\subsection{Modelos de Aprendizaje de Máquina en la Predicción del Tráfico}
Algoritmos de Aprendizaje de Máquina (Machine Learning) como Random Forest,
Support Vector Machines (SVM) y XGBoost han sido utilizados para predecir la
congestión del tráfico con alta precisión, logrando mejoras del 35-40\% en la
estimación de tráfico en comparación con métodos tradicionales \cite{Li2024}.

\subsection{Aplicaciones en Ciudades Inteligentes}
En Verona, Italia, se ha implementado un sistema de IA para gestionar el tráfico,
clasificando vehículos y ajustando semáforos en tiempo real, mejorando la fluidez
del tráfico y reduciendo la congestión \cite{Euronews2024}.


\section{Descripción de la Tarea}
Este proyecto busca desarrollar un modelo inteligente para la predicción del nivel de
congestión del tráfico utilizando técnicas de Inteligencia Artificial (IA). Se basará en
datos históricos y modelos de aprendizaje automático para identificar patrones de
congestión en distintos escenarios urbanos.

\subsection{Objetivo Principal}
Implementar un modelo que prediga el nivel de congestión del tráfico utilizando
variables como velocidad vehicular, ocupación de la vía, número de vehículos, accidentes
y condiciones climáticas.

\subsection{Entradas del Sistema}
\begin{itemize}
    \item \textbf{Datos del tráfico:} Vehicle\_Count, Traffic\_Speed\_kmh, Road\_Occupancy\_\%.
    \item \textbf{Condiciones climáticas:} Weather\_Condition (lluvia, nieve, despejado, etc.).
    \item \textbf{Información geográfica:} Latitude, Longitude, para analizar tendencias espaciales de congestión.
    \item \textbf{Historial de accidentes:} Accident\_Report, para evaluar su impacto en los niveles de tráfico.
\end{itemize}

\subsection{Salida del Sistema}
El agente debe procesar los datos y generar predicciones, que pueden ser:
\begin{itemize}
    \item Estimación del nivel de tráfico en franjas horarias futuras.
    \item Identificación de patrones de congestión en días específicos.
    \item Análisis de tendencias (cómo cambia el tráfico a lo largo del año).
\end{itemize}

\subsection{Restricciones}
\begin{itemize}
    \item \textbf{Este agente no puede reaccionar a eventos en tiempo real} (como accidentes o desvíos
    inesperados), pero sí puede anticipar patrones recurrentes.
    \item \textbf{Dependencia de Datos Históricos:}
    \begin{itemize}
        \item El sistema se basará en datos previos para hacer predicciones, por lo que su
        precisión dependerá de la calidad y representatividad del dataset utilizado.
        \item El dataset tiene registros en intervalos de 5 minutos, lo que limita la capacidad del
        modelo para capturar fluctuaciones muy rápidas en la congestión del tráfico.
        \item La precisión de la predicción depende de la densidad de los datos disponibles en
        diferentes zonas urbanas, lo que puede generar resultados menos precisos en
        áreas con menor cantidad de registros históricos.
    \end{itemize}
    \item \textbf{Falta de Información sobre Eventos Externos:}
    \begin{itemize}
        \item No se podrán considerar accidentes, obras viales, desvíos inesperados o eventos
        especiales si no están reflejados en la data histórica.
        \item El modelo asumirá que los patrones del tráfico siguen comportamientos
        recurrentes.
    \end{itemize}
    \item \textbf{Modelo Basado en Predicción de Tendencias:}
    \begin{itemize}
        \item No podrá generar recomendaciones personalizadas en tiempo real para usuarios
        individuales.
        \item Solo estimará la probabilidad de congestión en ciertos períodos futuros, sin poder
        adaptarse dinámicamente a cambios inesperados.
    \end{itemize}
\end{itemize}

\subsection{Alcances}
\begin{itemize}
    \item \textbf{Identificación de Patrones de Congestión}
    \begin{itemize}
        \item Permitirá detectar tendencias de tráfico recurrentes, como horas pico y días con
        mayor carga vehicular.
        \item Puede ayudar a tomar decisiones estratégicas sobre movilidad con base a datos
        previos.
    \end{itemize}
    
    \item \textbf{Aplicabilidad en Ciudades Inteligentes}
    \begin{itemize}
        \item Puede integrarse en plataformas de análisis de movilidad urbana para facilitar la
        toma de decisiones basada en datos históricos.
    \end{itemize}
    
    \item \textbf{Generación de Reportes de Análisis}
    \begin{itemize}
        \item Puede entregar reportes con predicciones de tráfico en distintos horarios y días.
        \item Útil para planificación de rutas o estrategias de movilidad basadas en datos
        históricos.
    \end{itemize}
\end{itemize}


\section{Análisis de Potencialidades para el Uso de Herramientas de Inteligencia Artificial}
La técnica más utilizada en este ámbito son las Redes Neuronales Artificiales (ANNs). El
estudio realizado por Goenawan et al. (2024) presenta un modelo basado en Redes
Neuronales Convolucionales (CNN) y LSTM, alcanzando una precisión del 85\% en la
predicción de tráfico, superando los métodos estadísticos tradicionales \cite{Goenawan2024}.

Por otro lado, Li et al. (2024) analizaron modelos de Aprendizaje de Máquina como
Regresión Logística y Árboles de Decisión, concluyendo que los métodos basados en
Redes Neuronales Multicapa (MLPNN) presentan mejor rendimiento en la clasificación
del tráfico en comparación con modelos lineales \cite{Li2024}.

En otro estudio, Elsalamony et al. (2013) evaluaron el uso de Árboles de Decisión C5.0 en conjunto con Redes Neuronales
Perceptrón Multicapa (MLPNN), demostrando una mayor precisión en la clasificación de
tráfico congestionado.

Asimismo, Al-Shayea et al. (2013) propusieron el uso de Redes de Propagación
Retroalimentadas y concluyeron que estos modelos tienen mejor rendimiento en la
predicción de tráfico urbano al captar relaciones complejas entre variables como
velocidad vehicular, ocupación de la vía y número de vehículos.

Por otro lado, Liu et al. (2008) exploraron el uso de Algoritmos Genéticos (GA) para la
selección de características en la predicción de tráfico, logrando una mejora significativa
en la precisión del modelo al identificar las variables más relevantes que afectan la
congestión.

Finalmente, Khan et al. (2013) aplicaron un modelo basado en Lógica
Difusa, convirtiendo las variables continuas en categorías de congestión y logrando una
mejor clasificación de los niveles de tráfico urbano.

Con base en estos estudios, el presente proyecto utilizará una combinación de Redes
Neuronales, Algoritmos Evolutivos, Lógica Difusa y Aprendizaje de Máquina, con el
objetivo de desarrollar un modelo preciso para la predicción del nivel de congestión del
tráfico, optimizando la movilidad urbana y mejorando la planificación del tránsito.

\section{Análisis de Viabilidad de Validación Experimental}
El problema que abordamos en este proyecto es la clasificación del nivel de congestión del tráfico en tres categorías: bajo, medio o alto. Para garantizar que el modelo de predicción sea confiable y preciso, es fundamental establecer un proceso riguroso de validación experimental. 
\subsection{Selección del Conjunto de Datos}
Dado que contamos con un dataset histórico de tráfico, el proceso de validación se realizará dividiendo los datos en dos conjuntos principales: 
\begin{itemize}
    \item \textbf{Conjunto de Entrenamiento: } Se utilizará para ajustar los modelos de IA, identificando patrones en los datos históricos de congestión.
    \item \textbf{Conjunto de Prueba: } Permitirá evaluar el rendimiento del modelo en datos no vistos, asegurando su capacidad de generalización.
\end{itemize}

\subsection{Experimento}
Este experimento tiene como objetivo evaluar el impacto de diferentes técnicas de balanceo de datos y modelos de clasificación en la precisión de la predicción del nivel de congestión del tráfico. Se analizarán dos factores: la técnica de balanceo de datos (sin balanceo, SMOTE, ADASYN y Random Oversampling) y el modelo de clasificación (Redes Neuronales, Lógica Difusa y Algoritmos Genéticos), lo que genera un diseño factorial completo 4×3 con 12 combinaciones experimentales. La variable de respuesta será la precisión del modelo medida mediante F1-score, precisión, sensibilidad y especificidad. Cada combinación se probará en un conjunto de datos de prueba fijo con 5 repeticiones y validación cruzada de K=5. El experimento incluirá preprocesamiento y balanceo de datos, entrenamiento de modelos y evaluación de desempeño. Finalmente, se determinará qué técnica de balanceo y qué modelo ofrecen el mejor rendimiento.



\section{Bibliografía}
\begin{thebibliography}{99}

    \bibitem{CEPAL} Comisión Económica para América Latina y el Caribe (CEPAL): La congestión del tránsito: sus consecuencias económicas y sociales. CEPAL, 2003.

    \bibitem{IMCO} Instituto Mexicano para la Competitividad (IMCO): El costo de la congestión: vida y recursos perdidos. IMCO, 2017.

    \bibitem{Goenawan2024} Goenawan, C. R.: ASTM: Autonomous Smart Traffic Management System Using Artificial Intelligence CNN and LSTM. arXiv preprint arXiv:2410.10929, 2024.
    
    \bibitem{SmartMobility2024} Smart Mobility Reports: Optimización del tráfico mediante Algoritmos Evolutivos. Andina Link Smart Cities, 2024.
    
    \bibitem{TransportationScience2024} Transportation Science: Aplicaciones de Lógica Difusa en el control del tráfico. Interempresas, 2024.
  
    \bibitem{Li2024} Li, H., Zhao, Y., Mao, Z., et al.: Graph Neural Networks in Intelligent Transportation Systems: Advances, Applications and Trends. arXiv preprint arXiv:2401.00713, 2024.

    \bibitem{Euronews2024} Euronews: Verona prueba un sistema con IA para mejorar el tráfico y la seguridad vial. Euronews Next, 2024.
  
    \bibitem{Elsalamony2013} Elsalamony, A. H., et al.: Using Decision Tree C5.0 and Multilayer Perceptron Neural Networks for Traffic Congestion Prediction. Journal of Advanced Transportation, 2013.
    
    \bibitem{AlShayea2013} Al-Shayea, Q.: Artificial Neural Networks in Traffic Prediction. International Journal of Computer Science and Network Security (IJCSNS), 2013.
    
    \bibitem{Liu2008} Liu, Y., et al.: Using Genetic Algorithms for Feature Selection in Traffic Prediction Models. Transportation Research Record: Journal of the Transportation Research Board, 2008.
    
    \bibitem{Khan2013} Khan, M. J., et al.: Fuzzy Logic-Based Traffic Congestion Prediction System. International Journal of Intelligent Transportation Systems, 2013.
  
  \end{thebibliography}

\end{document}
